% notes.tex
\documentclass[11pt]{article}

\usepackage{amsmath}
\usepackage{amssymb}
\usepackage{longtable}

\author{Ada Augusta, Countess of Lovelace}
\title{Notes By the Translator Upon the Memoir: Sketch of the Analytical Engine Invented by Charles Babbage}
\date{October, 1842}
\begin{document}
\maketitle

\section{Note A}
% intro.tex
These cards contain within themselves (in a manner explained in the Memoir 
itself \cite{menabrea_sketch_1842}) the law of development of the particular 
function that may be under consideration, and they compel the mechanism to act
accordingly in a certain corresponding order.

The particular function whose integral the Difference Engine was constructed to
tabulate, is $\Delta^7u_x=0$.


\section{Note B}
% storehouse.tex
\label{sec:storehouse}
In fact the engine may be described as being the material expression
of any indefinite function of any degree of generality and complexity,
such as for instance,

\begin{equation}
F(x, y, z, \log x, \sin y, x^p)
\end{equation}


\section{Note C}
The following is a more complicated example of the manner in which the
engine would compute a trig function containing variables. 
To multiply

\begin{align}
&A+A_1 \cos \theta + A_2\cos 2\theta + A_3\cos 3\theta + ...
\intertext{by}
&B + B_1 \cos \theta.
\end{align}

\section{Note D}
% example.tex
We have represented the solution of these two equations below, with every detail, in a diagram similar to those used in Note \ref{sec:storehouse}; ...


\section{Note E}
\section{Note F}
\section{Note G}

\bibliographystyle{plain}
\bibliography{refs}
\end{document}
